% Copyright 2014-2016 Joakim Nilsson
%
% This text is free software: you can redistribute it and/or modify
% it under the terms of the GNU General Public License as published by
% the Free Software Foundation, either version 3 of the License, or
% (at your option) any later version.
%
% This text is distributed in the hope that it will be useful,
% but WITHOUT ANY WARRANTY; without even the implied warranty of
% MERCHANTABILITY or FITNESS FOR A PARTICULAR PURPOSE.  See the
% GNU General Public License for more details.
%
% You should have received a copy of the GNU General Public License
% along with this text.  If not, see <http://www.gnu.org/licenses/>.

\newcommand{\nonTrump}{\textnormal{icke-trumf-bud}}

\begin{table}
	\caption{Specialbud}\label{tab:specialBids}
	\begin{center}
		\begin{tabularx}{\textwidth}{lcp{3cm}|X}
			\textbf{Namn} &
			\textbf{Värde} &
			\textbf{Inkompatibilitet} &
			\textbf{Tilläggsregler}
			\\[-3ex]

			\specialBidItem%
			{Sengångare}
			{$-3$}
			{---}
			{%
				För de stick därvar spelföraren inte spelar ut, spelar spelföraren ut sist.
			}

			\specialBidItem%
			{Triumf-Trumf}
			{$-3$}
			{---}
			{%
				Spelföraren väljer ett valfritt kort innan spelet börjar. Detta kort blir \emph{triumf-trumfen}. Spelföraren bestämmer vem som tar sticket med trium-trumfen när detta stick tas hem. Triumf-trumfen hamnar \emph{inte} i trumffärgen utan behåller sin gamla färg.
			}

			\specialBidItem%
			{Potential}
			{$-2$}
			{---}
			{%
				Om budet går hem markeras det med ett P, en \emph{potential}, i spelförarens kolonn och denne blir \emph{potentiell}.
			}

			\specialBidItem%
			{Start}
			{$-2$}
			{---}
			{%
				Spelföraren spelar ut i första sticket.
			}

			\specialBidItem%
			{Skick}
			{$-1$}
			{---}
			{%
				Före spelet börjar skickar alla spelare $3$ kort i en riktning som spelföraren väljer (till höger, till vänster eller tvärs över).
			}

			\specialBidItem%
			{Järn}
			{$-1$}
			{---}
			{%
				Essen är nu de lägsta korten.
			}

			\specialBidItem%
			{Girighet}
			{$0$}
			{---}
			{%
				I slutet av spelet läggs $1$ stick till eller dras bort från spelföraren så att det missgynnar denne. Om budet går hem får spelföraren $1$ extra-poäng.
			}

			\specialBidItem%
			{Slut-Hund}
			{$1$}
			{Noll}
			{%
				Spelföraren får inte ta hem det sista sticket.
			}

			\specialBidItem%
			{Ateljé}
			{$1$}
			{Öppen Hand}
			{%
				Spelföraren väljer $4$ kort som denne lägger i \emph{ateljén}. Dessa kort visas till samtliga spelare under spelets gång. Så fort det inte längre finns $4$ kort i ateljen måste spelföraren lägga dit ett nytt kort från handen såvida detta är möjligt.
			}

			\specialBidItem%
			{Öppen Trumf}
			{$1$}
			{\nonTrump, Grill, Öppen Hand}
			{%
				Spelföraren måste spela med öppna trumfkort. Det vill säga, spelförarens trumfkort måste visas till samtliga spelare under spelets gång. Om detta bud kombineras med \emph{Ateljé} så får ateljén inte innehålla några trumfkort.
			}

			\specialBidItem%
			{Lås}
			{$2$}
			{Noll}
			{%
				Spelföraren får inte ta hem något av de $3$ första sticken.
			}

			\specialBidItem%
			{Straff}
			{$2$}
			{---}
			{%
				Om budet inte går hem så dras $2$ extra poäng bort från dennes poängsumma.
			}

			\specialBidItem%
			{Utökat Bud}
			{$2$}
			{---}
			{%
				Detta bud får endast bjudas om spelföraren har gått hem med specialbudet \emph{Potential}, det vill säga om spelföraren är potentiell. När det bjuds stryks ett P som spelföraren har samlat på sig genom att gå hem med \emph{Potential}. Flertalet \emph{Utökade Bud} får ingå i ett kombinations-bud.
			}

			\specialBidItem%
			{Mästarskick}
			{$3$}
			{\nonTrump}
			{%
				Före spelet börjar skickar alla utom spelföraren $4$ kort till spelaren till höger (spelföraren hoppas över). Om \emph{Skick} har bjudits så skickar \emph{Skick}-korten före \emph{Mästarskick}-korten.
			}

			\specialBidItem%
			{Öppen Hand}
			{$3$}
			{Ateljé, Öppen Trumf}
			{%
				Spelföraren måste spela med öppen hand. Det vill säga, alla dennes kort måste visas till samtliga spelare under spelets gång.
			}
		\end{tabularx}
	\end{center}
\end{table}
