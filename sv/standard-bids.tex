% Copyright 2014-2016 Joakim Nilsson
%
% This text is free software: you can redistribute it and/or modify
% it under the terms of the GNU General Public License as published by
% the Free Software Foundation, either version 3 of the License, or
% (at your option) any later version.
%
% This text is distributed in the hope that it will be useful,
% but WITHOUT ANY WARRANTY; without even the implied warranty of
% MERCHANTABILITY or FITNESS FOR A PARTICULAR PURPOSE.  See the
% GNU General Public License for more details.
%
% You should have received a copy of the GNU General Public License
% along with this text.  If nejt, see <http://www.gnu.org/licenses/>.

\begin{table}
	\caption{Standardbud}\label{tab:standardBids}
	\begin{center}
		\begin{tabularx}{\textwidth}{lcccc|X}
				\textbf{Namn} &
				\rotccw{\textbf{Värde}} &
				\rotccw{\textbf{Poäng}} &
				\rotccw{\textbf{Trumf}} &
				\rotccw{\textbf{Stick}} &
				\textbf{Tilläggsregler}
				\\[-3ex]

				\standardBidItem%
				{Dumskipp}
				{$0$}
				{$1$}
				{nej}
				{varierar}
				{%
					Spelföraren får inte ta hem flest stick. Om hen tar hem samma mängd som någon annan spelare går budet inte hem.
				}

				\standardBidItem%
				{Ungefär}
				{$1$}
				{$1$}
				{nej}
				{varierar}
				{%
					Före spelets början gissar spelföraren på två möjliga mängder stick denne kommer att ta hem. Hen måste ta hem en av de två möjliga mängderna som gissades.
				}

				\standardBidItem%
				{Trumf*}
				{$1$}
				{$1$}
				{ja}
				{min 5}
				{%
					Spelföraren bestämmer trumffärg.
				}

				\standardBidItem%
				{Grill*}
				{$1$}
				{$2$}
				{ja}
				{min 5}
				{%
					Spelföraren börjar med att bestämma trumffärg. Denna trumffärg gäller bara första sticket. Därefter blir den färg som spelades ut i föregångde stick ny trumffärg och så fortsätter det till spelets slut.
				}
				
				\standardBidItem%
				{Spader*}
				{$2$}
				{$1$}
				{ja}
				{min 5}
				{%
					Trumffärgen är spader.
				}

				\standardBidItem%
				{Spel*}
				{$2$}
				{$2$}
				{nej}
				{min 5}
				{%
					---
				}

				\standardBidItem%
				{Skipp*}
				{$3$}
				{$2$}
				{nej}
				{varierar}
				{%
					Spelföraren måste ta hem färst stick. Om ingen tar hem färre stick än spelföraren går budet hem.
				}

				\standardBidItem%
				{Exakt}
				{$3$}
				{$2$}
				{nej}
				{varierar}
				{%
					Före spelets början gissar spelföraren på en möjlig mängd stick denne kommer att ta hem. Hen måste ta hem den mängd som gissades.
				}

				\standardBidItem%
				{Maxtrumf*}
				{$3$}
				{$3$}
				{ja}
				{min 7}
				{%
					Spelföraren väljer trumffärg.
				}

				\standardBidItem%
				{Smygtrumf*}
				{$3$}
				{$3$}
				{ja}
				{min 5}
				{%
					Spelföraren väljer trumffärg. Hen får dock inte välja en trumffärg som hen har flest kort i.
				}

				\standardBidItem%
				{Mästarspel}
				{$4$}
				{$3$}
				{nej}
				{varierar}
				{%
					Spelföraren måste ta hem flest stick. Om denna tar hem lika många som nån annan spelare så går budet inte hem.
				}

				\standardBidItem%
				{Noll}
				{$4$}
				{$4$}
				{nej}
				{0}
				{%
					---
				}

				\standardBidItem%
				{Mästartrumf*}
				{$6$}
				{$6$}
				{ja}
				{min 5}
				{%
					Spelaren till vänster om spelföraren bestämmer trumffärg, men först får de andra icke-spelförarna säga vilken trumffärg de föredrar och hur mycket de föredrar denna på en skala från $1$ till $10$ (utan motivering).
				}

				\standardBidItem%
				{Obesudlat Mästarspel}
				{$9$}
				{X}
				{nej}
				{alla}
				{%
					Om budet går hem får spelföraren lika många poäng som kombinations-budets värde. Skulle dessutom kombinations-budets värde vara $13$ eller högre och resultera i att spelföraren vinner spelet omedelbart så erhåller denne rätten att titulera sig \emph{Obesudlad Mästare av Kombinations-Whist} under resten av sitt liv. Om spelföraren tar hem färre än hälften av sticken förlorar denne dubbelt som många poäng som hen annars skulle ha gjort.
				}
		\end{tabularx}
	\end{center}
\end{table}
